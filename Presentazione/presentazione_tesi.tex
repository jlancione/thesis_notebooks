\documentclass{beamer}

\usepackage[utf8]{inputenc}
%\usepackage[scaled]{uarial} % not available in Linux
\renewcommand{\rmdefault}{phv} % Arial

\renewcommand{\sfdefault}{phv} % Arial
\renewcommand*\familydefault{\sfdefault} %% Only if the base font of the document is to be sans serif

\usepackage[T1]{fontenc}

\title{Classificazione dei bosoni elettrodeboli con una rete neurale al Large Hadron Collider} 
\subtitle{7 Novembre 2024}
\date{November 7, 2024}
\author{Jacopo Lancione}

\usetheme{FAIR}
    %\setbeamercovered{invisible} %default
    \setbeamercovered{transparent} % Items to be uncovered already visible, but almost transparent
    
\begin{document}
\begin{frame}
    \titlepage
\end{frame}

\begin{frame}{Sommario}
    \tableofcontents
\end{frame}


\section{LHC}
\begin{frame}{Large Hadron Collider}
    This is a Text
\end{frame}

\section{Machine Learning}
\begin{frame}{Machine Learning}
  \begin{columns}
    \column{.4\linewidth}
%   \vspace*{3ex}
    \begin{block}{}
      \centering
      \textbf{
      Unsupervised\\[.5ex]
      Learning}
    \end{block}
    \vspace*{2ex}
    \begin{itemize}
      \item Clustering
      \item Dimensional reduction
    \end{itemize}

    \column{.4\linewidth}
    \begin{block}{}
      \centering
      \textbf{
      Supervised\\[.5ex]
      Learning}
    \end{block}
    \vspace*{2ex}
    \begin{itemize}
      \item Classification
      \item other
    \end{itemize}
  \end{columns}

\end{frame}

\subsection{Reti Neurali}
\begin{frame}{Reti Neurali}
  neuroni/nodi

  layer

  pesi e bias

  attivazione che introduce la nn linearità 


\end{frame}



\begin{frame}{Conclusioni}
  qua ci metteremo i risultati
\end{frame}

\end{document}
