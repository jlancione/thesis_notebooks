\documentclass{beamer}

\usepackage[utf8]{inputenc}
%\usepackage[scaled]{uarial} % not available in Linux
\renewcommand{\rmdefault}{phv} % Arial

\renewcommand{\sfdefault}{phv} % Arial
\renewcommand*\familydefault{\sfdefault} %% Only if the base font of the document is to be sans serif

\usepackage[T1]{fontenc}

\title{Classificazione dei bosoni elettrodeboli con una rete neurale al Large Hadron Collider} 
\subtitle{7 Novembre 2024}
\date{November 7, 2024}
\author{Jacopo Lancione}

\usetheme{FAIR}
    %\setbeamercovered{invisible} %default
    \setbeamercovered{transparent} % Items to be uncovered already visible, but almost transparent
    
\begin{document}
\begin{frame}
    \titlepage
\end{frame}

\begin{frame}{Sommario}
    \tableofcontents
\end{frame}


\section{Introduzione}
\subsection{LHC}
\begin{frame}{Large Hadron Collider}
  L'anello di accelerazione + grande del cern (foto del logo), parliamo del CMS da cui arrivano i miei dati, in cui si producono particelle in abbondanza,

  (foto del cern e dl cms)

  Del CMS mi interessa giusto introdurre il fatto che ci siano dei calorimetri perché alcuni loro parametri sono tra le feature

  Si produce un enorme mole di dati e per trattarli si utilizzano anche tecniche di machine learning (e così passo alla prox slide)

  E posso accennare molto rapidamente al Nobel di quest'anno
\end{frame}

\subsection{Machine Learning}
\begin{frame}{Machine Learning}
  In questa slide devo far passare il concetto di labeled e unlabeled data (posso anche metterle come item) e la separazione del dataset in train test e validation (per il caso supervised)
  \begin{columns}[T]
    \column{.4\linewidth}
%   \begin{minipage}{\textwidth}
      \vspace*{3ex}
      \begin{block}{}
        \centering
        \textbf{
        Unsupervised\\[.5ex]
        Learning}
      \end{block}
      \vspace*{1ex}
%     {\centering Unlabeled data}
      \vspace*{1ex}
      \begin{itemize}
        \item Clustering
        \item Riduzione dimensionale
      \end{itemize}
%   \end{minipage}
%
    \column{.4\linewidth}
%   \begin{minipage}{\textwidth}
      \vspace*{3ex}
      \begin{block}{}
        \centering
        \textbf{
        Supervised\\[.5ex]
        Learning}
      \end{block}
      \vspace*{1ex}
%     {\centering Labeled data}
      \vspace*{1ex}
      \begin{itemize}
        \only<1>{
        \item Classificazione
        }
        \only<2>{
        \item {\color{faircolor}CLASSIFICAZIONE}
        }
        \item Riconoscimento di immagini e testo
      \end{itemize}
%   \end{minipage}
  \end{columns}
  \vspace{3ex}
  Qua dicendo che il mio progetto ruota attorno ad un problema di classificazione passo alla prox slide
\end{frame}

\begin{frame}{Il Progetto di tesi}
% \vspace{-8ex}
% \begin{center}
%   \color{faircolor}\Large\bfseries Il Progetto di tesi
% \end{center}
% \vspace{5ex}
  Che sia chiaro dove si colloca il mio progetto:
  affrontare un problema di classificazione binaria (logistic regression), nell'ambito della Fisica delle alte energie

  Immagine classica del modello std e 1 di 1 rete neurale giusto per dire rapidamente il Cosa e il Come

  il mio obiettivo: allenare 1 rete che distingua al meglio tra i 2 canali di decadimento

\end{frame}

\section{Dataset}
\begin{frame}
  \centering
  \Huge\bfseries
  Il Dataset
\end{frame}

\subsection{Decadimenti dei bosoni}
\begin{frame}{Decadimenti di $Z$ e $W$}
  Diciamo subito qlche dettaglio in $+$ sul dataset (e magari mettiamolo anche a fondo slide, il riferimento a dove ho scaricato i dati)

  qua posso mettere i diagrammi di Feynman dei decadimenti, giusto per mettere qualcosa sotto gli occhi al pubblico

  Elencare le features ie la cinematica di interesse e anche le variabili lasciate da parte in riferimento al rivelatore

% Ql è in breve la fisica del processo
\end{frame}

\subsection{Preprocessing}
\begin{frame}{Preprocessing}
  Racconto di come ho trattato i dati: la storia del chi2 (vogliamo dirla? nn ne conosco i dettagli purtroppo), la qstione dgli outlier

  Qua mostriamo sicuramente i pairplot che sono la cosa più indicativa, magari anche i boxplot? -> in questa maniera escono $+$ slides (e qua posso sprecarmi con il logaritmo e lo questione del'approccio scartato con le sigma)
  
  Raccontiamo la storia di correlazioni evidenti che permetterebbero una facile classificazione, nel caso $+$ semplice attraverso 1 appl lineare

  I concetti da far passare sono 2: è meglio è avere 1 dataset uniforme, quindi scaliamo tutto e ci sbarazziamo degli outlier, evitare di introdurre ridondanze (ie guardare in faccia i dati con pairplot)
\end{frame}

\section{Reti Neurali}
\subsection{Architettura e principi}
\begin{frame}
  \centering
  \Huge\bfseries
  Reti Neurali
\end{frame}

\begin{frame}{L'architettura}
  % sicuramente la platea apprezza la concisione ie qnte info riesco a far passare nel minor tempo a patto di 1 buona qltà dl'info
  Qua bisogna introdurre i parametri su cui ho agito: numero di layer, nodi, attivazione, algoritmo di ottimizzazione (questo nella slide successiva)

  E devo spiegare come funziona la questione dei parametri (pesi e bias) e dove si introduce la non linearità (attivazione)

  Questa slide la organizzerei come un elenco .ntato a sx e una bella immagine con cui io riesca a spiegare tutto 
\end{frame}

\begin{frame}{Loss function \\Algoritmi di ottimizzazione}
  % Qsto per spiegare come funzioni la rete, nn per mostrare risultati ancora
  L'idea di 1 loss function da minimizzare (come se fosse un'energia), qsta nn è semplice da spiegare visivamente, di questa metterei proprio la formula così la commento un attimo

  Algoritmo (sarebbe carino accennare al learning rate e al momento e all'adattività degli algoritmi)


  Immagini di allenamenti significativi, magari anche in cui si veda l'overfitting

\end{frame}

\subsection{Risultati dell'allenamento}
\begin{frame}{Risultati}
  E qua ci va una carrellata di rock curves che può tranquillamente occupare $+$ slides, quali voglio scegliere come significative? Con algoritmi diversi e mostrando bene il test point

\end{frame}



\begin{frame}{Conclusioni}
  \begin{itemize}
    \item Le reti neurali si prestano molto bene a compiti di particle identification
    \item Sono degli strumenti molto flessibili e quindi il mio progetto è facilmente generalizzabile ad altre necessità/misure
    \item Ho identificato una classe di modelli equivalenti
  \end{itemize}
  \vspace{2ex}

  studi futuri: provare a combinare i dataset e allenare una rete su quelli per distinguere i bosoni uno dall'altro

    \vspace*{4ex}
    \begin{columns}
      \column{.4\linewidth}
      {
        \only<1>{\setbeamercolor{block body}{bg=white, fg=white}}
        \begin{block}{}
          \centering\vspace*{1ex}
          \only<1>{}
          \only<2>{
            \Large\bfseries
            \color{white}
            Grazie a tutti!%
          }%
        \vspace*{1ex}
      \end{block}
      }
    \end{columns}
\end{frame}

\end{document}
